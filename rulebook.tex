% -------------------------------------------------------------------
% Author: Aman Rathore
% Contact: amanr.me | amanrathore9753 <at> gmail <dot> com
% Created on: Wednesday, November 12, 2025 at 12:19
% -------------------------------------------------------------------
\documentclass[11pt,a4paper]{article}

\usepackage[margin=1in]{geometry}
\usepackage{enumitem}
\usepackage{array,tabularx,booktabs,longtable}
\usepackage[dvipsnames]{xcolor}
\usepackage{hyperref}
\usepackage{fancyhdr}
\usepackage{lastpage}
\usepackage{datetime2}
\usepackage[T1]{fontenc}
\usepackage{palatino}

\hypersetup{
	colorlinks=true,
	linkcolor=MidnightBlue,
	urlcolor=MidnightBlue
}

\pagestyle{fancy}
\fancyhf{}
\lhead{IISER Mohali Robotics Club}
\rhead{Club Room Rulebook}
\cfoot{\thepage\ of \pageref{LastPage}}

\setlist[itemize]{topsep=2pt,itemsep=2pt,parsep=0pt}
\setlist[enumerate]{topsep=4pt,itemsep=4pt,parsep=0pt}

\newcommand{\Club}{IISER Mohali Robotics Club}
\newcommand{\Room}{Robotics Club Room}
\newcommand{\Version}{v0.1}
%\newcommand{\EffectiveDate}{\DTMdisplaydate{2025}{11}{12}{-1}}
\newcommand{\EffectiveDate}{Currently in review}
\newcommand{\ClubEmail}{\href{mailto:roboticsclub@iisermohali.ac.in}{roboticsclub@iisermohali.ac.in}}

\begin{document}
	
	\begin{center}
 {\LARGE \Club: Club Room Rulebook}\\[4pt]
 \Version\ \textbar\ Effective: \EffectiveDate
	\end{center}
	
	\paragraph{Purpose}
	This rulebook defines safe and fair use of the \Room\ for projects, storage, fabrication, and testing. It applies to all members, guests, and collaborators.
	
	% ============================================================
	\section*{Roles, Categories \& Access Credentials}
	
	\begin{tabularx}{\linewidth}{p{3.2cm} X p{3.5cm} p{3.5cm}}
 \toprule
 Category & Who & Access scope & Credentials \\
 \midrule
 Members &
 Students working with the club on a temporary or short-term basis (e.g., end-sem PHY211 project, course minis, one-off tasks). &
 Access during posted hours or with prior permission; may use tools per policy under supervision where required. &
 No biometric. Sign-in required. \\
 \addlinespace[3pt]
 Active members (biometric) &
 Ongoing contributors to official club projects and operations. &
 Access during posted hours; may open/close the room as designated; may supervise permitted operations per policy. &
 Biometric eligible. Assigned only after approval by club officials; revocable. \\
 \addlinespace[3pt]
 Guests / visitors &
 Not working on any official club project (drop-ins, friends, visitors, outreach). &
 Entry only when accompanied by a permitted member or allowed official; no tool or equipment operation. Observation only. &
 No biometric. Temporary entry only. \\
 \addlinespace[3pt]
 Club officials (convenors and ex-convenors) &
 Appointed/previously appointed student leads responsible for operations, safety, and stewardship. &
 May approve access, schedule operational holds, and supervise restricted tools; ex-convenors act as escalation for complaints. &
 Manage biometric assignment for active members; maintain email-first records. \\
 \bottomrule
	\end{tabularx}
	
	\medskip
	\noindent\textit{Biometric policy.} Biometric access is assigned only to \emph{active members} after approval by club officials and can be revoked for safety, misuse, or inactivity. “Members” are temporary contributors (e.g., PHY211/end-sem projects) and are not eligible for biometric by default.
	
	\newpage
	
	% ============================================================
	\section*{A. Access \& Conduct}
	\begin{enumerate}
 \item Access Hours. Use the club only during posted access hours or with prior permission from allowed officials/permitted members in charge; biometric entry applies to approved \emph{active members} only.
 \item Operational Holds. The club may temporarily stop access for segregation of components, audits, cleaning, or maintenance. Such holds will be announced by email when possible.
 \item Night Work (Recommendation). Heavy/noisy or hazardous operations (drilling, grinding, extensive soldering) are discouraged at night. Aim to finish such operations before 22:00.
 \item Sign-In/Out. Record your name, roll number, designation (Member / Active member / Guest / Official), in/out time, and purpose in the log book if available.
 \item Guests. Guests/visitors need prior permission, must be accompanied by a permitted member, and may not use tools or equipment.
 \item Respect. Treat people, tools, and ongoing projects with care. No harassment, vandalism, or disruptive behavior.
	\end{enumerate}
	
	% ============================================================
	\section*{B. Food, Drinks \& Cleanliness}
	\begin{enumerate}
 \item Food. Dry snacks are fine. Keep liquids far from electronics and machines.
 \item Clean Up. Clear waste immediately; wipe benches you used.
 \item Spills. If a spill occurs, stop work, disconnect affected gear (if safe), and email the club regarding the incident with full transparency.
	\end{enumerate}
	
	% ============================================================
	\section*{C. Storage \& Labelling}
	\begin{enumerate}
 \item Project Bins on Tables Only. Keep project bins/boxes only on the tables, not even on the standing desk at the center. Do not place project bins in the almira.
 \item Almira Use (Sensitive/Expensive Projects). Prior permission from allowed officials is required to use the almira for expensive/sensitive projects.
 \item Labels. Each bin must list project name and/or name of at least one team member.
 \item Clearing. Unattended/unclear items may be cleared after notice.
	\end{enumerate}
	
	% ============================================================
	\section*{D. Components: Taking, Returning, Reporting}
	\begin{enumerate}
 \item Take Only What You Need. Be conservative with passives and small parts.
 \item Return To Correct Boxes. If you used a resistor from the resistor box, put it back in the correct value drawer/strip. Same applies to transistors, diodes, ICs, headers, jumpers, dev-boards, sensors, motors, drivers, screws, nuts, washers. Keep assortments tidy.
 \item If Unsure. Don’t dump misc parts into random bins. Ask an allowed official or leave them in a clearly labelled “to-sort” tray, which will be available on the standing desk at the center.
 \item Damage/Missing. If a part/tool is damaged or missing, email \ClubEmail\ immediately. If you found it damaged, mention when it was last seen in good condition (date/time/person if known).
	\end{enumerate}
	
	% ============================================================
	\section*{E. Tool \& Equipment Policy}
	\subsection*{E1. Permissions Matrix}
	\renewcommand{\arraystretch}{1.2}
	\begin{tabularx}{\linewidth}{>{\raggedright\arraybackslash}p{4.2cm} >{\raggedright\arraybackslash}X >{\raggedright\arraybackslash}p{3.6cm}}
 \toprule
 Category & Use Policy & Notes \\
 \midrule
 Hand tools (screwdrivers, pliers, cutters, tweezers) &
 Members and Active members may use. Guests/visitors may not operate tools. &
 Return to correct peg/slot; report issues by email. \\
 Soldering \& small electronics tools &
 Members and Active members may use. Guests/visitors may not operate tools. &
 Use fume extraction if available; do not solder near the 3D printer. \\
 Power tools (hand drill, rotary/angle grinder) &
 Only allowed officials or permitted Active members may operate; Members require explicit permission and supervision; Guests/visitors may not operate. &
 PPE mandatory (glasses, mask/ear where applicable). \\
 Oscilloscope, logic analyzer, function generator &
 Members/Active members only; use only if you know correct operation or with supervision. Guests/visitors may not operate. &
 Wrong settings can damage probes/instruments. \\
 3D printer &
 No direct use without permission. Requests only via email to \ClubEmail. Operation by allowed officials or permitted members. Guests/visitors may not operate. &
 Strictly scientific/project use; no personal prints. \\
 \bottomrule
	\end{tabularx}
	
	\subsection*{E2. General Rules}
	\begin{enumerate}
 \item Keep Hazards Away From 3D Printer. Do not use drills, grinders, or soldering anywhere near the 3D printer.
 \item Night Work (Recommendation). Avoid drilling/grinding/extended soldering at night; target 22:00 as a soft cutoff for such operations.
 \item Power Off. After use, switch off instruments and unplug portable tools.
	\end{enumerate}
	
	
	% ============================================================
	\section*{F. 3D Printing}
	\begin{enumerate}
 \item Channel. All 3D print requests go by email to \ClubEmail.
 \item Printer and materials. Printer: Bambu Lab A1. Filament stocked: PLA, PLA+, Pla Pro (all the filaments are subject to availability).
 Maximum build volume: 25.6\,cm $\times$ 25.6\,cm $\times$ 25.6\,cm;
 recommended working envelope: up to 24\,cm cube for reliability.
 \item Allowed use. Only scientific/academic project work. Personal or non-scientific items (door hooks, generic holders, etc.) are not allowed.
 \item Operation. Printing is performed or supervised by allowed officials or permitted members. No unsupervised operation. Guests/visitors may not request or operate prints.
 \item Models and files. Email your STL/OBJ (or other exported model files) to club along with your name, project name, and a brief reason/justification for the print. Also specify the filament type and colour you want. The Robotics Club will review the request, may ask for changes or decline/cancel prints that are costly, unsafe, or irrelevant, and (if approved) will return a club-sliced file for the Bambu Lab A1 (typically .gcode.3mf). For the step-by-step workflow (request $\rightarrow$ approval $\rightarrow$ sending to printer), see \hyperref[app:3dprinter]{Appendix~A}.
 \item Safety near the printer. Do not drill, grind, or solder anywhere near the printer. Keep the surrounding bench clear; do not touch the machine without permission.
	\end{enumerate}
	
	% ============================================================
	\section*{G. Equipment Use Duration \& Project Email (No Hard Limits)}
	\begin{enumerate}
 \item No Fixed Time Limits. There is no formal time limit on equipment use. Be considerate so others can proceed.
 \item Mandatory Project Email. For each project, send an email to \ClubEmail\ including:
 \begin{itemize}
 \item Detailed list of all parts/components you used (ICs, sensors, motors, passives, hardware, PCBs/boards, etc.).
 \item Short project description (goal, current status).
 \item Assistance log: which allowed officials/permitted members assisted you in using restricted tools (e.g., 3D printer, angle grinder), including date(s).
 \end{itemize}
	\end{enumerate}
	
	% ============================================================
	\section*{H. Electrical \& Test Safety}
	\begin{enumerate}
 \item Know Your Limits. If unsure about scope probes, PSU limits, or grounding—pause and ask a club official.
 \item Voltage/Current Limits. Set current limits before powering circuits. Double-check polarity and ratings.
 \item Probes. Use appropriate probes/ranges. Never exceed rated voltage.
	\end{enumerate}
	
	% ============================================================
	\section*{I. Housekeeping \& Lock-up}
	\begin{enumerate}
 \item Before Leaving:
 \begin{itemize}
 \item Power down instruments; unplug portable tools; close laptops.
 \item Return all tools/components to proper places; clear the bench; dispose of trash.
 \end{itemize}
 \item Doors \& Windows. Close and secure all the doors and windows, including terrace entry.
 \item Lights \& Fans. Switch off lights and fans.
	\end{enumerate}
	
	% ============================================================
	\section*{J. Contacts}
	Primary club email for all operational communication: \ClubEmail
	
	\medskip
	Current officials:
	\begin{itemize}
 \item Aman Rathore — \href{mailto:ms24058@iisermohali.ac.in}{ms24058@iisermohali.ac.in} \quad Phone: \texttt{+91 80034 79539}
 \item Thoihen Yendrembam — \href{mailto:ms24202@iisermohali.ac.in}{ms24202@iisermohali.ac.in} \quad Phone: \texttt{+91 98628 20031}
 \item Argho Ghughu — \href{mailto:ms24180@iisermohali.ac.in}{ms24180@iisermohali.ac.in} \quad Phone: \texttt{+91 62910 48847}
 \item Harsh Vardhan Shreshth — \href{mailto:ms24085@iisermohali.ac.in}{ms24085@iisermohali.ac.in} \quad Phone: \texttt{+91 93899 72664}
 \item Kshitij Pravin Salunke — \href{mailto:ms24024@iisermohali.ac.in}{ms24024@iisermohali.ac.in} \quad Phone: \texttt{+91 80973 08023}
 \item Tamaghna Dey — \href{mailto:ms24205@iisermohali.ac.in}{ms24205@iisermohali.ac.in} \quad Phone: \texttt{+91 98745 17714}
	\end{itemize}
	
	
	
	% ============================================================
	\clearpage
	\section*{Appendix A. Using the 3D Printer (Bambu Lab A1): Full Protocol}
	\label{app:3dprinter}
	
	This appendix provides the operational workflow referenced in Section~F (3D Printing). The intent is to keep slicing settings consistent and safe for the club's standard materials, especially the locally sourced Rajendra's generic PLA, and numakers' PLA+ \& PLA Pro.
	
	\subsection*{A1. Submit a print request by email (STL/OBJ only)}
	\begin{enumerate}
 \item Email your exported model file(s) (.stl, .obj, or equivalent) to \ClubEmail.
 \item In the same email, clearly include:
 \begin{itemize}
 \item Your name and contact details.
 \item Your project name (club/campus project) and a short reason/justification for the print (why it is needed).
 \item Requested filament type and colour. If you are unsure, specify the reason in detail so that club officials can choose the best option.
 \item Quantity and any deadlines (if applicable). Urgent requests may be declined if they disrupt scheduled club work.
 \end{itemize}
	\end{enumerate}
	
	\subsection*{A2. Club review and approval}
	\begin{itemize}
 \item The Robotics Club will review the request for feasibility, cost (time/filament), relevance to scientific/academic project work, and safety.
 \item If the print is costly, irrelevant, or requires changes (splitting parts, mouse ears, wall thickness, room for expansion, etc.), the club may request modifications or may decline/cancel the print. This is at the club's discretion.
	\end{itemize}
	
	\subsection*{A3. Receive the club-sliced file (do not re-slice)}
	\begin{enumerate}
 \item If approved, the club will reply with a club-sliced file intended for the Bambu Lab A1, typically a .gcode.3mf file.
 \item If your requested filament/colour is available, the reply will also specify the spool number / AMS slot to use.
 \item Multi-filament prints are not allowed at present. Use a single filament/spool per print.
 \item The reason the club provides the sliced file is to ensure consistent settings tuned for the club's common PLA variants. Do not modify the file or re-slice with custom settings; request changes by email instead.
	\end{enumerate}
	
	\subsection*{A4. Send the job to the printer (mobile or laptop)}
	\begin{enumerate}
 \item Install Bambu Lab's softare on your device.
 \item Log in using the Robotics Club printer account:
 \begin{itemize}
 \item The club email ID and password are written on the paper sheet taped on top of the 3D printer.
 \end{itemize}
 \item Connectivity:
 \begin{itemize}
 \item Connect to the club room Wi-Fi using the SSID/password, or use mobile data.
 \end{itemize}
 \item Upload and send the file:
 \begin{itemize}
 \item Upload the received .gcode.3mf file to the Bambu app, select the Bambu Lab A1 printer, and send the job.
 \item Confirm you are sending to the correct printer before starting.
 \end{itemize}
 \item Filament setup:
 \begin{itemize}
 \item Use the spool number / AMS slot indicated in the club's reply.
 \item Ensure the AMS slot in use matches the filament specified in the approval email.
 \end{itemize}
 \item Start the print and monitor till the first two layers are printed correctly. Do not leave the club room until the printer goes to the 3rd layer.
	\end{enumerate}
	
	\subsection*{A5. If something needs to change}
	If the print fails, looks incorrect, or you need a design or slicing change, do not experiment with new slicing profiles. Contact robotics club, club will advise or re-slice as needed.
	
\end{document}